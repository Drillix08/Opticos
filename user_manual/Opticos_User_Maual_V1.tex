\documentclass{article}
\usepackage{geometry, graphicx} % Required for inserting images

\title{Opticos User Manual}
\author{David Green}
\date{October 2025}

\begin{document}

\maketitle

\section*{Introduction}
Hello and welcome to the Opticos user manual! In this manual we will be going over wat Opticos is, the different features that the program offers, and how to run each individual feature.

\section*{What is Opticos?}
Opticos is educational math tool with the goal of visually demonstrating complex math topics using precise computer animations. The name Opticos come from the greek word "optikos", which means visual. When we say that Opticos uses animations, we don't mean the kind used as children's entertainment. We mean the type of animations used in computer simulations, in which objects are moved in a precise way to visualize complex ideas.
\bigskip

Furthermore, the motivation behind Opticos is that there is lack of modern accesible tools to visualize higher level math concepts. The computer has come such a long way and yet when it comes to math education, we still rely almost exclusively on messy hand drawings on white boards or chalkboards to visualize concepts. With Opticos, we aim to provide an additional supplementary tool to help visually explain math concepts in a clearer and more complete way. We want Opticos to be a useful resource for students who want to learn and a useful resource for instructors who want to help better explain concepts.

\section*{The Two Features of Opticos}
Opticos has two different features, the \textbf{Interactive Animator} and \textbf{the Opticos Textbook}. The interactive animator is an tool that ask you to input a function and a set of other input values which are then used to create an animation demonstrating a particular concept. The Opticos Textbook is a mini textbook with standard text explanations of math concepts but with a little twist. Accompanying the text explanation are pre-rendered animations to help support said explanations. Let's talk more in detail about each of these two features.

\end{document}
